\documentclass[]{book}
\usepackage{lmodern}
\usepackage{amssymb,amsmath}
\usepackage{ifxetex,ifluatex}
\usepackage{fixltx2e} % provides \textsubscript
\ifnum 0\ifxetex 1\fi\ifluatex 1\fi=0 % if pdftex
  \usepackage[T1]{fontenc}
  \usepackage[utf8]{inputenc}
\else % if luatex or xelatex
  \ifxetex
    \usepackage{mathspec}
  \else
    \usepackage{fontspec}
  \fi
  \defaultfontfeatures{Ligatures=TeX,Scale=MatchLowercase}
\fi
% use upquote if available, for straight quotes in verbatim environments
\IfFileExists{upquote.sty}{\usepackage{upquote}}{}
% use microtype if available
\IfFileExists{microtype.sty}{%
\usepackage{microtype}
\UseMicrotypeSet[protrusion]{basicmath} % disable protrusion for tt fonts
}{}
\usepackage[margin=1in]{geometry}
\usepackage{hyperref}
\hypersetup{unicode=true,
            pdfborder={0 0 0},
            breaklinks=true}
\urlstyle{same}  % don't use monospace font for urls
\usepackage{natbib}
\bibliographystyle{apalike}
\usepackage{color}
\usepackage{fancyvrb}
\newcommand{\VerbBar}{|}
\newcommand{\VERB}{\Verb[commandchars=\\\{\}]}
\DefineVerbatimEnvironment{Highlighting}{Verbatim}{commandchars=\\\{\}}
% Add ',fontsize=\small' for more characters per line
\usepackage{framed}
\definecolor{shadecolor}{RGB}{248,248,248}
\newenvironment{Shaded}{\begin{snugshade}}{\end{snugshade}}
\newcommand{\KeywordTok}[1]{\textcolor[rgb]{0.13,0.29,0.53}{\textbf{#1}}}
\newcommand{\DataTypeTok}[1]{\textcolor[rgb]{0.13,0.29,0.53}{#1}}
\newcommand{\DecValTok}[1]{\textcolor[rgb]{0.00,0.00,0.81}{#1}}
\newcommand{\BaseNTok}[1]{\textcolor[rgb]{0.00,0.00,0.81}{#1}}
\newcommand{\FloatTok}[1]{\textcolor[rgb]{0.00,0.00,0.81}{#1}}
\newcommand{\ConstantTok}[1]{\textcolor[rgb]{0.00,0.00,0.00}{#1}}
\newcommand{\CharTok}[1]{\textcolor[rgb]{0.31,0.60,0.02}{#1}}
\newcommand{\SpecialCharTok}[1]{\textcolor[rgb]{0.00,0.00,0.00}{#1}}
\newcommand{\StringTok}[1]{\textcolor[rgb]{0.31,0.60,0.02}{#1}}
\newcommand{\VerbatimStringTok}[1]{\textcolor[rgb]{0.31,0.60,0.02}{#1}}
\newcommand{\SpecialStringTok}[1]{\textcolor[rgb]{0.31,0.60,0.02}{#1}}
\newcommand{\ImportTok}[1]{#1}
\newcommand{\CommentTok}[1]{\textcolor[rgb]{0.56,0.35,0.01}{\textit{#1}}}
\newcommand{\DocumentationTok}[1]{\textcolor[rgb]{0.56,0.35,0.01}{\textbf{\textit{#1}}}}
\newcommand{\AnnotationTok}[1]{\textcolor[rgb]{0.56,0.35,0.01}{\textbf{\textit{#1}}}}
\newcommand{\CommentVarTok}[1]{\textcolor[rgb]{0.56,0.35,0.01}{\textbf{\textit{#1}}}}
\newcommand{\OtherTok}[1]{\textcolor[rgb]{0.56,0.35,0.01}{#1}}
\newcommand{\FunctionTok}[1]{\textcolor[rgb]{0.00,0.00,0.00}{#1}}
\newcommand{\VariableTok}[1]{\textcolor[rgb]{0.00,0.00,0.00}{#1}}
\newcommand{\ControlFlowTok}[1]{\textcolor[rgb]{0.13,0.29,0.53}{\textbf{#1}}}
\newcommand{\OperatorTok}[1]{\textcolor[rgb]{0.81,0.36,0.00}{\textbf{#1}}}
\newcommand{\BuiltInTok}[1]{#1}
\newcommand{\ExtensionTok}[1]{#1}
\newcommand{\PreprocessorTok}[1]{\textcolor[rgb]{0.56,0.35,0.01}{\textit{#1}}}
\newcommand{\AttributeTok}[1]{\textcolor[rgb]{0.77,0.63,0.00}{#1}}
\newcommand{\RegionMarkerTok}[1]{#1}
\newcommand{\InformationTok}[1]{\textcolor[rgb]{0.56,0.35,0.01}{\textbf{\textit{#1}}}}
\newcommand{\WarningTok}[1]{\textcolor[rgb]{0.56,0.35,0.01}{\textbf{\textit{#1}}}}
\newcommand{\AlertTok}[1]{\textcolor[rgb]{0.94,0.16,0.16}{#1}}
\newcommand{\ErrorTok}[1]{\textcolor[rgb]{0.64,0.00,0.00}{\textbf{#1}}}
\newcommand{\NormalTok}[1]{#1}
\usepackage{longtable,booktabs}
\usepackage{graphicx,grffile}
\makeatletter
\def\maxwidth{\ifdim\Gin@nat@width>\linewidth\linewidth\else\Gin@nat@width\fi}
\def\maxheight{\ifdim\Gin@nat@height>\textheight\textheight\else\Gin@nat@height\fi}
\makeatother
% Scale images if necessary, so that they will not overflow the page
% margins by default, and it is still possible to overwrite the defaults
% using explicit options in \includegraphics[width, height, ...]{}
\setkeys{Gin}{width=\maxwidth,height=\maxheight,keepaspectratio}
\IfFileExists{parskip.sty}{%
\usepackage{parskip}
}{% else
\setlength{\parindent}{0pt}
\setlength{\parskip}{6pt plus 2pt minus 1pt}
}
\setlength{\emergencystretch}{3em}  % prevent overfull lines
\providecommand{\tightlist}{%
  \setlength{\itemsep}{0pt}\setlength{\parskip}{0pt}}
\setcounter{secnumdepth}{5}
% Redefines (sub)paragraphs to behave more like sections
\ifx\paragraph\undefined\else
\let\oldparagraph\paragraph
\renewcommand{\paragraph}[1]{\oldparagraph{#1}\mbox{}}
\fi
\ifx\subparagraph\undefined\else
\let\oldsubparagraph\subparagraph
\renewcommand{\subparagraph}[1]{\oldsubparagraph{#1}\mbox{}}
\fi

%%% Use protect on footnotes to avoid problems with footnotes in titles
\let\rmarkdownfootnote\footnote%
\def\footnote{\protect\rmarkdownfootnote}

%%% Change title format to be more compact
\usepackage{titling}

% Create subtitle command for use in maketitle
\newcommand{\subtitle}[1]{
  \posttitle{
    \begin{center}\large#1\end{center}
    }
}

\setlength{\droptitle}{-2em}

  \title{}
    \pretitle{\vspace{\droptitle}}
  \posttitle{}
    \author{}
    \preauthor{}\postauthor{}
    \date{}
    \predate{}\postdate{}
  
\usepackage{booktabs}
\usepackage{booktabs}
\usepackage{longtable}
\usepackage{array}
\usepackage{multirow}
\usepackage[table]{xcolor}
\usepackage{wrapfig}
\usepackage{float}
\usepackage{colortbl}
\usepackage{pdflscape}
\usepackage{tabu}
\usepackage{threeparttable}
\usepackage{threeparttablex}
\usepackage[normalem]{ulem}
\usepackage{makecell}

\usepackage{amsthm}
\newtheorem{theorem}{Theorem}[chapter]
\newtheorem{lemma}{Lemma}[chapter]
\theoremstyle{definition}
\newtheorem{definition}{Definition}[chapter]
\newtheorem{corollary}{Corollary}[chapter]
\newtheorem{proposition}{Proposition}[chapter]
\theoremstyle{definition}
\newtheorem{example}{Example}[chapter]
\theoremstyle{definition}
\newtheorem{exercise}{Exercise}[chapter]
\theoremstyle{remark}
\newtheorem*{remark}{Remark}
\newtheorem*{solution}{Solution}
\begin{document}

{
\setcounter{tocdepth}{1}
\tableofcontents
}
\chapter{Introduction}\label{intro}

This documentation describes the ``news intensity'' dashboard and API.
These are tools for gauging the amount of relevant media coverage for
any asset. That is to say, our news intensities (NI) measure the volume
of news stories that are realistically pertinent to an asset, regardless
of whether they mention that asset by name. For example, rare earths are
used in making smartphones, therefore a story about rare-earth mining is
potentially of relevance to a company manufacturing smartphones, whether
or not the company is mentioned in the story. Likewise, Peru is a leader
in zinc mining, therefore a story about political turmoil in Peru may
have an impact on the prices of zinc futures, whether or not the
connection is made in the story itself.

The notion that asset prices are impacted by the nature and extent of
media coverage (even when it does not contain new information) has found
strong support in the finance literature (e.g.
\citet{huberman2001contagious}, \citet{engelberg2011causal},
\citet{tetlock2011all}). By capturing the amount of related media
coverage for any asset, NI are expected to have incremental explanatory
power for asset price movements.

The calculation of NI relies on Cortical.io's multi-lingual semantic
fingerprinting technology, which ingests text and outputs its ``semantic
fingerprint'' in the form of a sparse 128x128 matrix of 0s and 1s. Each
position in the matrix represents a topic, so that the fingerprint of a
text indicates which of the possible 16,384 topics the text is related
to. Every day and for every supported language, we obtain the aggregate
semantic fingerprint of online news stories from thousands of sources.
The count of the relevant stories for each topic is then normalized so
that the average normalized value across topics is equal to 1. We also
obtain semantic fingerprints for traded assets based on their textual
descriptions (\citet{ibriyamova2017using},
\citet{ibriyamova2018predicting} show that such fingerprints have
significant information content). The value of NI for any
day/language/asset combination is the average of the normalized article
counts across all topics related to the asset in question.

\chapter{Getting started}\label{gettinstarted}

This section contains general information about querying companies and
commodities for both the API and the dashboard.

If you have been given an API key, visit the
\href{https://fintxt.github.io/documentation/theapi.html}{API} section.
If you have been given access to the FinTxt dashboard, visit the
\href{https://fintxt.github.io/documentation/thedashboard.html}{dashboard}
section.

\section{Languages}\label{languages}

FinTxt news intensity values are based on news articles. These articles
are written in either one of the following languages:

\begin{enumerate}
\def\labelenumi{\arabic{enumi}.}
\tightlist
\item
  English
\item
  French
\item
  German
\item
  Arabic
\item
  Russian
\end{enumerate}

When you retrieve news intensities, you can either select one of these
languages or opt to use the aggregate of all languages (`total').

\section{Selecting companies}\label{selecting-companies}

When using the API or the dashboard, you should query companies by using
their
\href{https://en.wikipedia.org/wiki/Reuters_Instrument_Code}{Reuters
Instrument Code}.

\section{Selecting commodities}\label{selecting-commodities}

You can query the API and dashboard for either companies or commodities
(but not a mix of both). The following commodities are available:

\begin{itemize}
\tightlist
\item
  corn
\item
  oats
\item
  rice\\
\item
  soybeans\\
\item
  rapeseed\\
\item
  wheat\\
\item
  milk\\
\item
  cocoa\\
\item
  coffee\\
\item
  cotton\\
\item
  sugar\\
\item
  oranges\\
\item
  ethanol\\
\item
  propane\\
\item
  copper\\
\item
  lead\\
\item
  zinc\\
\item
  tin\\
\item
  aluminium
\item
  nickel\\
\item
  cobalt\\
\item
  molybdenum
\item
  steel\\
\item
  gold\\
\item
  platinum
\item
  palladium
\item
  silver
\item
  rubber
\item
  wool
\item
  amber
\end{itemize}

\chapter{News Intensity API}\label{theapi}

The FinTxt news intensity API is hosted at
\url{https://api.fintxt.io/rest}. You can view the endpoints and their
documentation \href{https://api.fintxt.io/rest/__swagger__/}{here}.

The API contains four endpoints:

\begin{enumerate}
\def\labelenumi{\arabic{enumi}.}
\tightlist
\item
  \textbf{languages}: returns a list of available language for which you
  can query news intensities
\item
  \textbf{live}: returns live news intensity metrics for a company or
  commodity and a language
\item
  \textbf{historic}: returns historic news intensity metrics for a
  company or commodity, a language and a date
\item
  \textbf{portfolio}: returns live or historic news intensity metrics
  for a weighted portfolio of stocks or commodities, a language and a
  date.
\end{enumerate}

The \emph{language} endpoint can be queried without an API key. The
\emph{historic} endpoint can be queried without an API key for dates
that go back beyond 30 days on the date of today.

\section{Basic usage}\label{basic-usage}

View the \href{https://api.fintxt.io/rest/__swagger__/}{documentation}
for each endpoint for more information about headers and post bodies.

Essentially, you can query each endpoint by building the URL. For
example, you want to query the news intensity value for the commodity
\texttt{wool} on 04-06-2018 for texts written in arabic. Your query
would then look as follows:

\begin{verbatim}
languages = 'arabic'
date = '04-06-2018'
endpoint = 'historic'
type = 'commodities'
q = 'wool'
\end{verbatim}

You could construct this request using \texttt{curl}:

\begin{Shaded}
\begin{Highlighting}[]
\ExtensionTok{curl}\NormalTok{ -X GET }\StringTok{"https://api.fintxt.io/rest/historic/commodities/arabic/04-06-2018?q=wool"}\NormalTok{ -H  }\StringTok{"accept: application/json"}
\end{Highlighting}
\end{Shaded}

You could use \texttt{httr} in R:

\begin{Shaded}
\begin{Highlighting}[]
\NormalTok{languages <-}\StringTok{ 'arabic'}
\NormalTok{date <-}\StringTok{ '04-06-2018'}
\NormalTok{endpoint <-}\StringTok{ 'historic'}
\NormalTok{type <-}\StringTok{ 'commodities'}
\NormalTok{q <-}\StringTok{ 'wool'}

\NormalTok{url <-}\StringTok{ }\KeywordTok{paste0}\NormalTok{(}\StringTok{"https://api.fintxt.io/rest/"}\NormalTok{, endpoint, }\StringTok{"/"}\NormalTok{, type, }\StringTok{"/"}\NormalTok{, languages, }\StringTok{"/"}\NormalTok{, date, }\StringTok{"?q="}\NormalTok{, q)}

\NormalTok{resp <-}\StringTok{ }\NormalTok{httr}\OperatorTok{::}\KeywordTok{GET}\NormalTok{(url)}
\NormalTok{httr}\OperatorTok{::}\KeywordTok{content}\NormalTok{(resp)}
\end{Highlighting}
\end{Shaded}

Or you could use \texttt{requests} in Python:

\begin{Shaded}
\begin{Highlighting}[]
\ImportTok{import}\NormalTok{ json}
\ImportTok{import}\NormalTok{ requests}

\NormalTok{languages }\OperatorTok{=} \StringTok{'arabic'}
\NormalTok{date }\OperatorTok{=} \StringTok{'04-06-2018'}
\NormalTok{endpoint }\OperatorTok{=} \StringTok{'historic'}
\NormalTok{type_ }\OperatorTok{=} \StringTok{'commodities'}
\NormalTok{q }\OperatorTok{=} \StringTok{'wool'}

\NormalTok{url }\OperatorTok{=} \StringTok{"https://api.fintxt.io/rest/}\SpecialCharTok{\{\}}\StringTok{/}\SpecialCharTok{\{\}}\StringTok{/}\SpecialCharTok{\{\}}\StringTok{/}\SpecialCharTok{\{\}}\StringTok{?q=}\SpecialCharTok{\{\}}\StringTok{"}\NormalTok{.}\BuiltInTok{format}\NormalTok{(endpoint, type_, languages, date, q)}

\CommentTok{# Send request }
\NormalTok{r }\OperatorTok{=}\NormalTok{ requests.get(url)}

\CommentTok{# Load response}
\NormalTok{c }\OperatorTok{=}\NormalTok{ json.loads(r.content)}

\BuiltInTok{print}\NormalTok{(c)}
\end{Highlighting}
\end{Shaded}

But it would be easier to use the clients described below.

\section{R and Python clients}\label{r-and-python-clients}

You can use the R and Python 3 clients to retrieve data from the API.
These clients can be installed from their GitHub repositories.

\subsection{R client}\label{r-client}

Install the \href{https://github.com/FinTxt/FinTxtClient-R}{R client} by
executing

\begin{Shaded}
\begin{Highlighting}[]
\NormalTok{devtools}\OperatorTok{::}\KeywordTok{install_github}\NormalTok{(}\StringTok{"FinTxt/FinTxtClient"}\NormalTok{)}
\end{Highlighting}
\end{Shaded}

After installing the package, register your API token by calling the
following code:

\begin{Shaded}
\begin{Highlighting}[]
\KeywordTok{library}\NormalTok{(FinTxtClient)}
\KeywordTok{Sys.setenv}\NormalTok{(}\StringTok{"FINTXT_CLIENT_TOKEN"}\NormalTok{ =}\StringTok{ "<yourtoken>"}\NormalTok{)}
\end{Highlighting}
\end{Shaded}

You can now access the various endpoints:

\begin{Shaded}
\begin{Highlighting}[]
\CommentTok{# Set some variables}
\NormalTok{identifiers =}\StringTok{ }\KeywordTok{c}\NormalTok{(}\StringTok{"TRI.TO"}\NormalTok{, }\StringTok{"IBM.N"}\NormalTok{, }\StringTok{"RRD.N"}\NormalTok{, }\StringTok{"SPGI.N"}\NormalTok{, }\StringTok{"INTU.OQ"}\NormalTok{, }\StringTok{"RELN.AS"}\NormalTok{, }\StringTok{"WLSNc.AS"}\NormalTok{, }\StringTok{"REL.L"}\NormalTok{)}
\NormalTok{weights =}\StringTok{ }\KeywordTok{c}\NormalTok{(}\FloatTok{0.3}\NormalTok{, }\FloatTok{0.1}\NormalTok{,}\FloatTok{0.05}\NormalTok{,}\FloatTok{0.05}\NormalTok{, }\FloatTok{0.2}\NormalTok{,}\FloatTok{0.1}\NormalTok{,}\FloatTok{0.1}\NormalTok{,}\FloatTok{0.1}\NormalTok{)}
\NormalTok{date =}\StringTok{ "09-07-2018"}
\NormalTok{type =}\StringTok{ "companies"}
\NormalTok{language =}\StringTok{ "english"}

\CommentTok{# Load the client}
\KeywordTok{library}\NormalTok{(FinTxtClient)}

\CommentTok{# Call the languages endpoint}
\NormalTok{langs <-}\StringTok{ }\KeywordTok{fintxt_get_languages}\NormalTok{()}

\CommentTok{# Get the live intensity for a stock}
\NormalTok{one <-}\StringTok{ }\KeywordTok{fintxt_live_intensities_one}\NormalTok{(}\DataTypeTok{type =}\NormalTok{ type, }\DataTypeTok{language =}\NormalTok{ language, }\DataTypeTok{q =}\NormalTok{ identifiers[}\DecValTok{1}\NormalTok{])}
\CommentTok{# Same but for commodity}
\NormalTok{one <-}\StringTok{ }\KeywordTok{fintxt_live_intensities_one}\NormalTok{(}\DataTypeTok{type =} \StringTok{"commodities"}\NormalTok{, }\DataTypeTok{language =}\NormalTok{ language, }\DataTypeTok{q =} \StringTok{"milk"}\NormalTok{)}

\CommentTok{# Get historic intensity for a stock}
\NormalTok{one <-}\StringTok{ }\KeywordTok{fintxt_historic_intensities_one}\NormalTok{(}\DataTypeTok{type =}\NormalTok{ type, }\DataTypeTok{language =}\NormalTok{ language,}
                                       \DataTypeTok{date =}\NormalTok{ date, }\DataTypeTok{q=}\NormalTok{identifiers[}\DecValTok{1}\NormalTok{])}
                                       
\CommentTok{# Same but for commodity}
\NormalTok{one <-}\StringTok{ }\KeywordTok{fintxt_historic_intensities_one}\NormalTok{(}\DataTypeTok{type =} \StringTok{"commodities"}\NormalTok{, }\DataTypeTok{language =}\NormalTok{ language,}
                                       \DataTypeTok{date =}\NormalTok{ date, }\DataTypeTok{q=}\StringTok{"milk"}\NormalTok{)}

\CommentTok{# Get live intensity for a portfolio}
\NormalTok{port <-}\StringTok{ }\KeywordTok{fintxt_live_intensities_portfolio}\NormalTok{(}\DataTypeTok{type =}\NormalTok{ type,}
                                          \DataTypeTok{language =}\NormalTok{ language,}
                                          \DataTypeTok{identifiers =} \KeywordTok{c}\NormalTok{(identifiers, }\StringTok{"monkey"}\NormalTok{),}
                                          \DataTypeTok{weights =} \KeywordTok{c}\NormalTok{(weights, }\FloatTok{0.4}\NormalTok{))}
\CommentTok{# For commodity}
\NormalTok{port <-}\StringTok{ }\KeywordTok{fintxt_live_intensities_portfolio}\NormalTok{(}\DataTypeTok{type =} \StringTok{"commodities"}\NormalTok{,}
                                          \DataTypeTok{language =}\NormalTok{ language,}
                                          \DataTypeTok{identifiers =} \KeywordTok{c}\NormalTok{(}\StringTok{"milk"}\NormalTok{, }\StringTok{"soybeans"}\NormalTok{),}
                                          \DataTypeTok{weights =} \KeywordTok{c}\NormalTok{(}\FloatTok{0.5}\NormalTok{, }\FloatTok{0.5}\NormalTok{))}

\CommentTok{# Get historic intensity for a portfolio}
\NormalTok{port <-}\StringTok{ }\KeywordTok{fintxt_historic_intensities_portfolio}\NormalTok{(}\DataTypeTok{type =}\NormalTok{ type,}
                                          \DataTypeTok{language =}\NormalTok{ language,}
                                          \DataTypeTok{date =}\NormalTok{ date,}
                                          \DataTypeTok{identifiers =} \KeywordTok{c}\NormalTok{(identifiers, }\StringTok{"monkey"}\NormalTok{),}
                                          \DataTypeTok{weights =} \KeywordTok{c}\NormalTok{(weights, }\FloatTok{0.4}\NormalTok{))}
                                          
\CommentTok{# For commodity}
\NormalTok{port <-}\StringTok{ }\KeywordTok{fintxt_historic_intensities_portfolio}\NormalTok{(}\DataTypeTok{type =} \StringTok{"commodities"}\NormalTok{,}
                                              \DataTypeTok{language =}\NormalTok{ language,}
                                              \DataTypeTok{date =}\NormalTok{ date,}
                                              \DataTypeTok{identifiers =} \KeywordTok{c}\NormalTok{(}\StringTok{"milk"}\NormalTok{, }\StringTok{"soybeans"}\NormalTok{),}
                                              \DataTypeTok{weights =} \KeywordTok{c}\NormalTok{(}\FloatTok{0.5}\NormalTok{, }\FloatTok{0.5}\NormalTok{))}
\end{Highlighting}
\end{Shaded}

\subsection{Python client}\label{python-client}

To install the \href{https://github.com/FinTxt/FinTxtClient-Py}{python
client}, execute the following:

\begin{Shaded}
\begin{Highlighting}[]
\NormalTok{pip install git}\OperatorTok{+}\NormalTok{https:}\OperatorTok{//}\NormalTok{github.com}\OperatorTok{/}\NormalTok{FinTxt}\OperatorTok{/}\NormalTok{FinTxtClient}\OperatorTok{-}\NormalTok{Py.git }\OperatorTok{--}\NormalTok{user}
\end{Highlighting}
\end{Shaded}

You can import the package as follows:

\begin{Shaded}
\begin{Highlighting}[]
\ImportTok{from}\NormalTok{ FinTxtClient }\ImportTok{import}\NormalTok{ FinTxtClient}
\end{Highlighting}
\end{Shaded}

Then, you can initiate the client using:

\begin{Shaded}
\begin{Highlighting}[]
\NormalTok{client }\OperatorTok{=}\NormalTok{ FinTxtClient() }\CommentTok{# Optionally, pass 'key = <your-key-here>'}
\end{Highlighting}
\end{Shaded}

Using the client is as simple as calling the following functions:

\begin{Shaded}
\begin{Highlighting}[]
\CommentTok{# Call the languages endpoint}
\NormalTok{client.languages()}

\CommentTok{# Call the live endpoint for a commodity}
\NormalTok{client.live_one(}\StringTok{"commodities"}\NormalTok{, }\StringTok{"english"}\NormalTok{, }\StringTok{"milk"}\NormalTok{)}

\CommentTok{# Call the historic endpoint for a commodity and a date}
\NormalTok{client.historic_one(}\StringTok{"commodities"}\NormalTok{, }\StringTok{"english"}\NormalTok{, }\StringTok{"13-07-2018"}\NormalTok{, }\StringTok{"milk"}\NormalTok{)}

\CommentTok{# Call the live portfolio endpoint}
\NormalTok{client.live_portfolio( _type }\OperatorTok{=} \StringTok{"companies"}\NormalTok{, language }\OperatorTok{=} \StringTok{"english"}\NormalTok{, }
\NormalTok{                      identifiers}\OperatorTok{=}\NormalTok{[}\StringTok{"TRI.TO"}\NormalTok{, }\StringTok{"IBM.N"}\NormalTok{, }\StringTok{"RRD.N"}\NormalTok{, }\StringTok{"SPGI.N"}\NormalTok{, }\StringTok{"INTU.OQ"}\NormalTok{, }\StringTok{"RELN.AS"}\NormalTok{, }\StringTok{"WLSNc.AS"}\NormalTok{, }\StringTok{"REL.L"}\NormalTok{], }
\NormalTok{                      weights}\OperatorTok{=}\NormalTok{[}\FloatTok{0.3}\NormalTok{, }\FloatTok{0.1}\NormalTok{,}\FloatTok{0.05}\NormalTok{,}\FloatTok{0.05}\NormalTok{, }\FloatTok{0.2}\NormalTok{,}\FloatTok{0.1}\NormalTok{,}\FloatTok{0.1}\NormalTok{,}\FloatTok{0.1}\NormalTok{])}

\CommentTok{# Call the historic portfolio endpoint                    }
\NormalTok{client.historic_portfolio( _type }\OperatorTok{=} \StringTok{"companies"}\NormalTok{, language }\OperatorTok{=} \StringTok{"english"}\NormalTok{, date}\OperatorTok{=}\StringTok{"13-07-2018"}\NormalTok{, }
\NormalTok{                          identifiers}\OperatorTok{=}\NormalTok{[}\StringTok{"TRI.TO"}\NormalTok{, }\StringTok{"IBM.N"}\NormalTok{, }\StringTok{"RRD.N"}\NormalTok{, }\StringTok{"SPGI.N"}\NormalTok{, }\StringTok{"INTU.OQ"}\NormalTok{, }\StringTok{"RELN.AS"}\NormalTok{, }\StringTok{"WLSNc.AS"}\NormalTok{, }\StringTok{"REL.L"}\NormalTok{], }
\NormalTok{                          weights}\OperatorTok{=}\NormalTok{[}\FloatTok{0.3}\NormalTok{, }\FloatTok{0.1}\NormalTok{,}\FloatTok{0.05}\NormalTok{,}\FloatTok{0.05}\NormalTok{, }\FloatTok{0.2}\NormalTok{,}\FloatTok{0.1}\NormalTok{,}\FloatTok{0.1}\NormalTok{,}\FloatTok{0.1}\NormalTok{])}
\end{Highlighting}
\end{Shaded}

\chapter{News Intensity Dashboard}\label{thedashboard}

The news intensity dashboard is hosted at
\href{https://dashboard.fintxt.io/app/FinTxtDashboard}{https://dashboard.fintxt.io}
and is only accessible if you have received a username / password
combination. Before you read on, you should check out the
\href{https://fintxt.github.io/documentation/gettinstarted.html}{`getting
started' section}.

\section{Company overview}\label{company-overview}

After logging in to the dashboard, you can view the submenu's for the
`Company' section by click on the \texttt{Company} tab. Enter a
commodity name or company RIC into the search bar and select one of the
sub categories.

\begin{figure}
\centering
\includegraphics{img/company_overview.gif}
\caption{Querying a company or commodity}
\end{figure}

\section{Portfolio overview}\label{portfolio-overview}

To work with the portfolio overview, you first have to upload a
portfolio. This file has to be comma-delimited (CVS).

Your file must contain two columns; the first column must be called
`identifier' (this is where the company RICs or commodity names go) and
the second column must be called `weight' (this is where you specify the
percentage in decimal form).

An example of a portfolio file is given below.

\begin{table}[H]
\centering
\begin{tabular}{l|r}
\hline
identifier & weight\\
\hline
TRI.TO & 0.33\\
\hline
AAPL.OQ & 0.33\\
\hline
GOOGL.OQ & 0.33\\
\hline
\end{tabular}
\end{table}

Next, upload your portfolio using the `upload' functionality.

\begin{figure}
\centering
\includegraphics{img/createportfolio.gif}
\caption{Creating and uploading a portfolio}
\end{figure}

You can now use the portfolio features as shown below.

\begin{figure}
\centering
\includegraphics{img/portfolio_overview.gif}
\caption{Using the portfolio features}
\end{figure}

Note that you cannot mix both commodities and companies in your
portfolio. If you try this, you will see the following error:

\begin{figure}
\centering
\includegraphics{img/mixcomms.gif}
\caption{Mixing commodities and companies}
\end{figure}

\chapter{Resources}\label{resources}

// TODO: add David's research

\bibliography{book.bib,packages.bib}


\end{document}
